\chapter{Conclusion}
\label{chap:conclusion}
Le développement d'un nouveau médicament est un processus long et couteux. Entre la détermination d'une cible thérapeutique et la mise sur le marché d'un nouveau médicament, plus d'un milliard d'euros et de dix ans de recherches sont nécessaire.
L'accélération de ce processus et donc la réduction de son coût reste donc un enjeux majeur actuel. Pour y parvenir, les simulations numériques, peu couteuses et rapides, sont massivement utilisées. Malgré cela, elles restent limitées à cause de la quantité très importante de molécule de solvant.


La théorie de la fonctionnelle de la densité moléculaire permet d'étudier la solvatation de composés de n'importe qu'elle taille et de n'importe qu'elle forme. Elle permet en quelques secondes seulement d'obtenir à la fois l'énergie libre de solvatation et une carte détaillé de la densité d'équilibre autour de ce soluté.
Ces grandeurs étant à la base de nombreux autres calculs utilisées par l'industrie pharmaceutique, la MDFT ouvre donc une autre voie (XXX) d'optimisation de ces process.


Durant ma thèse, mon travail à consister à effectuer les premiers pas vers des applications biologiques. Cette thèse c'est déroulée en trois étapes majeures. La première consistait à adapter la théorie à des macro-molécules biologiques. Pour cela, nous avons développé une version à symmétrie de la Théorie de la Fonctionnelle de la Densité Moléculaire. Cette version, simplifiée et plus rapide, nous à permi de paramétriser un nouveau bridge: le bridge gros grain.
Ce bridge, basé sur une densité gros grain ajoute de la consistance thermodynamique à nos modèles, en reproduisant une pression et une tension de surface correcte. Il améliore également fortement le calcul de l’énergie libre de solvatation et la prédiction de la structure du solvant, sur les petites molécules mais également sur des molécules plus grosses comme des systèmes biologiques.


La seconde étape consistait à l'adapation du code. En effet, l'étude de composés de plusieurs milliers d'atomes à fait naître différentes limites techniques. Afin de dépasser ces limites, nous avons du adapter, optimiser et parralléliser le code MDFT.


Enfin, la dernière étape consistait à évaluer nos développements théoriques et numériques sur des systèmes d'interêts biologiques. Pour cela, trois études ont été menées. La première: le benchmarke de MDFT sur un ensemble de 604 copmposés de type médicament. Nous avons ainsi mis en évidence que la correction de pression \textit{PC+} n'est plus aujourd'hui adaptée à la théorie au niveau HNC. La meilleure précision dans la prédiction de l'energie libre de solvation de composés de type médicaments est obtenue à l'aide de la théorie dans l'approximation HNC couplée à la correction de pression \textit{PC} et pour une valeur de $\mathrm{m}_\mathrm{max}$=3.
Les deux autres applications ont permis d'évaluer MDFT sur des systèmes biologiques plus importants. Nous avons ainsi montré qu'il est possible avec MDFT (i) d’étudier avec précision des systèmes biologiques, (ii) de retrouver les molécules d’eau expérimentales et les poches à l’intérieur de protéines et (iii) d’améliorer la prédiction d’énergie libre de liaison en solution tout en fournissant la structure de solvatation. 

\boitesimple{En conclusion, durant cette thèse, nous avons adapté la théorie MFDT et son code associé afin de permettre une étude rapide et précise de systèmes biologiques. L'ensemble de ces travaux constituent un premier pas et ouvre une nouvelle voie d'application pour MDFT: la recherche de médicament.}






\chapter{Perspectives}
\label{chap:perspectives}
Les résultats obtenus durant cette thèse sont très encourageants. Cependant, MDFT reste une voie ouverte de recherche qui offre encore de nombreuses possibilités et il reste encore beaucoup à faire. Nous proposons ici un ensemble de perspectives non exhaustives qui viennent s'ajouter à celles déjà connues de MDFT sur les petits composés.

\section{MM-MDFT: l'approximation de trajectoire unique }
Les premiers résultats obtenus dans le cadre de la dérivation de MM-PBSA en MM-MDFT sont trés encourageants mais restent limités. En effet, l'utilisation de MDFT permet actuellement uniquement l'apport d'informations supplémentaires au travers de la structure du solvant. La précision ainsi que le temps de calcul restent les mêmes pour ces deux méthodes. Dans cette étude préliminaire, nous n'avons considéré qu'une seule conformation par complexe, la conformation après minimisation de la structure cristallographique. Pour aller plus loin et ainsi augmenter la précision des résultats obtenus, il est possible de calculer une trajectoire de dynamique moléculaire ou Monte Carlo du complexe dans le vide et d'en extraire différentes conformations à intervalle régulier. L'énergie libre de solvatation du système correspond, dans ce cas, à la moyenne des énergies libre de solvatation calculées pour chaque conformation. Cette technique, nommé approximation de trajectoire unique est connue pour fortement améliorer les résultats obtenus. Au moment de la rédaction de ce manuscrit, ces calculs sont en cours.

\section{Une étude plus complète des ions}
Le benchmark de MDFT sur des composés de type médicament, et plus particulièrement sur les ions, à permis de mettre en évidence une limite importante de MDFT: les charges partielles. Jusqu'ici nous avons uniquement étudié quelques ions monovalents. Les premiers résultats semblent indiquer que l'erreur pourrait dépendre uniquement de la charge du composé. Afin de confirmer cette tendance et ainsi mieux comprendre et donc corriger ce defaut, la prochaine étape indispensable est l'étude d'une base de données d'ions plus importante non restraintes à des composés monovalents.

\section{Machine learning}
Le développement de l'outil \textit{MDFT Database Tool} permet une obtention rapide, simple et automatique de résultats sur des bases de données de plusieurs milliers de composés. Ce développement à ainsi permi la création de banque de données de références indispensables à la mise en place d'outils de \textit{machine learning}. Face à l'augmentation exponentielle du nombre de données disponibles, les outils de \textit{machine learning} semblent inévitables. Les premiers résultats obtenus par Sohvi Luukkonen sont d'ailleurs trés encourageants.

\section{Couplage de MDFT}
Enfin, cette thèse à ouvert une nouvelle voie pour MDFT: la recherche pharmaceutique. Suite à ces développements, plusieurs applications directes sont actuellement envisageables. Les étapes suivantes consistent donc: (i) à coupler MDFT avec les techniques existantes (docking, virtual screening, ...). La substitution dans ces logiciels d'un solvant implicite par MDFT permettrait l'apport d'informations moléculaires supplémentaires pour des temps de calculs similaires. Et (ii) proposer des alternatives plus précises à certains calculs basés sur l'énergie libre de solvatation comme le calcul du logP ou encore celui du logBB.









