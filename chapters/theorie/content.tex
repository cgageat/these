\chapter{MDFT: la théorie de la fonctionnelle de la densité moléculaire}
\label{chap:theorie}

\boitemagique{Objectif}{
Dans ce chapitre nous décrivons la théorie de la fonctionnelle de la densité moléculaire dans l'approximation HNC, son implémentation ainsi que quelques corrections de cette approximation.
}


La théorie de la fonctionnelle de la densité moléculaire (MDFT) permet l'étude de la solvatation de composés de n'importe quelle taille et n'importe quelle forme à l'échelle moléculaire. Cette théorie et son code associé, permettent, en quelques secondes seulement, (i) de calculer l'énergie libre de solvatation et (ii) de générer une carte détaillée en 3 dimensions de la densité ainsi que de l'orientation du solvant autour du soluté.


L'origine de la théorie de la fonctionnelle de la densité (DFT) réside dans le développement d'une fonctionnelle $\mathcal{F}[\rho\left(\boldsymbol{r},\Omega \right)]$. Cette fonctionnelle a pour variable la densité du solvant $\rho\left(\boldsymbol{r},\Omega \right)$, en chaque point de l'espace $\boldsymbol{r}$ et pour chaque orientation $\Omega$ de la molécule de solvant. La fonctionnelle est construite comme la différence entre le grand potentiel du soluté en solution et le grand potentiel du solvant homogène de densité $\rho_{0}$. Par définition, la valeur de la fonctionnelle au minimum correspond donc à l'énergie libre de solvatation du soluté étudié. 

Sans approximation pour le moment, la fonctionnelle est découpée en trois parties: la partie idéale, la partie extérieure et la partie d'excès\cite{evans_density_2009,henderson_fundamentals_1992}. 
\begin{eqnarray}
\mathcal{F} = \mathcal{F}_\mathrm{id} + \mathcal{F}_\mathrm{ext} + \mathcal{F}_\mathrm{exc}
\label{eq:fonctionnelle}
\end{eqnarray}
La partie idéale, représente l’entropie d'information du système, et s'écrit
\begin{eqnarray}
\mathcal{F}_\mathrm{id}&=&\mathrm{k_B}T\int\mathrm{d}\boldsymbol{r}\mathrm{d}\Omega \rho\left(\boldsymbol{r},\Omega \right)\ln\left(\frac{\rho\left(\boldsymbol{r},\Omega \right)}{\rho_0}\right)-\Delta\rho\left(\boldsymbol{r},\Omega \right)
\label{eq:fonctionnelle:id}
\end{eqnarray}
\noindent avec $T$ la température, $\mathrm{k_B}$ la constante de Boltzmann et donc $\mathrm{k_B}T$ l'énergie thermique, $\Delta\rho\left(\boldsymbol{r},\Omega \right)=\rho\left(\boldsymbol{r},\Omega \right)-\rho_0$ la densité d'excès par rapport à la densité bulk de référence $\rho_0$. La seconde partie, la partie extérieure, représente le potentiel d'interaction $\phi\left(\boldsymbol{r},\Omega \right)$ entre le soluté et le solvant. Elle s'écrit:
\begin{eqnarray}
\mathcal{F}_\mathrm{ext}&=&\int\mathrm{d}\boldsymbol{r}\mathrm{d}\Omega\rho\left(\boldsymbol{r},\Omega \right)\phi\left(\boldsymbol{r},\Omega \right)
\label{eq:fonctionnelle:ext}
\end{eqnarray}
\noindent avec
\begin{eqnarray}
\phi\left(\boldsymbol{r},\Omega \right) = \sum\limits_{i=1}^{\mathrm{n}_\mathrm{sv}}\sum\limits_{j=1}^{\mathrm{n}_\mathrm{su}} v_{ij}(|\boldsymbol{r}+\boldsymbol{S}_i(\Omega)+\boldsymbol{r}_j|)
\end{eqnarray}
\noindent avec $\mathrm{n}_\mathrm{sv}$ le nombre de sites du solvent, $\mathrm{n}_\mathrm{su}$ le nombre de sites du solute et $\boldsymbol(S)_i$ le vecteur reliant l'origine du solvant au site i. Le potentiel d'interaction correspond à la somme des interactions électrostatiques et des interactions de Lennard-Jones ou toute autre interaction potentielle. 

\section{L'approximation HNC}

Enfin, la partie d'excès correspond à la corrélation entre les molécules de solvant. La version exacte de cette partie, correspond au développement de Taylor infini autour de la densité bulk liquide de référence, soit pour l'eau $\rho_0$=1kg.L$^{-1}$. Afin d'en permettre son calcul, des approximations doivent être considérées. Nous considérons ici uniquement le premier et le second ordres du développement:
\begin{eqnarray}
\mathcal{F}_\mathrm{exc} = -\frac{\mathrm{k_B}T}{2}\int\mathrm{d}\boldsymbol{r}\mathrm{d}\Omega \Delta\rho\left(\boldsymbol{r}, \Omega   \right) \gamma \left(\boldsymbol{r},\Omega\right)  + \mathcal{O}(\Delta\rho^{3})
\label{eq:fonctionnelle:exc}
\end{eqnarray}
\noindent avec
\begin{eqnarray}
\gamma \left(\boldsymbol{r},\Omega\right) = \int\mathrm{d}\boldsymbol{r}^\prime\mathrm{d}\Omega^\prime\  c\left(\boldsymbol{r}-\boldsymbol{r}^\prime,\Omega,\Omega^\prime \right) \Delta\rho\left(\boldsymbol{r}^\prime, \Omega^\prime \right)
\end{eqnarray}
\noindent avec $c\left(\boldsymbol{r}-\boldsymbol{r}^\prime,\Omega,\Omega^\prime \right) $ la fonction de corrélation directe entre deux molécules de solvant qui dépend de la distance entre ces molécules et de leurs orientations relatives l'une par rapport à l'autre. La fonction de corrélation directe pour un solvant homogène à température et pression données est issue de longues simulations de dynamique moléculaire ou de Monte Carlo corrigées des effets de taille finie\cite{Puibasset_bridge_2012, belloni_unpublished}.

Cette approximation, bien connue, est nommée approximation HNC (hyper-Netted Chain)\cite{hansen_theory_2006}. 


\section{L'implémentation}
Une fois la fonctionnelle décrite, il est nécessaire de la minimiser. En effet, par définition, le minimum de la fonctionnelle correspond à l'énergie libre de solvatation. Dans le même temps, le minimum est atteint lorsqu'en tout point de l'espace, la densité du solvant équivaut à sa densité dite à l'équilibre.
\begin{eqnarray}
\min(\mathcal{F}[\rho\left(\boldsymbol{r},\Omega \right)]) = \mathcal{F}[\rho_{eq}\left(\boldsymbol{r},\Omega \right)])= \Delta G_{solv}
\end{eqnarray}
Cette minimisation a été implémentée dans un code en Fortran moderne du même nom: MDFT.


\subsection{La discrétisation}
Comme il n'est numériquement pas possible de travailler avec un système continu infini, le soluté est étudié dans un système fini, discret et périodique. Il existe deux niveaux de discrétisation du système. Le premier, spatial, découpe l'espace sur une grille homogène. Le second niveau, angulaire\cite{ding_cea-01564512}, permet de limiter le nombre d'orientations étudiées. Il est actuellement possible de choisir entre vitesse et précision en faisant varier le nombre d'angles étudiés de 18 à 726 à travers un paramètre nommé $\mathrm{m}_\mathrm{max}$. L'équivalence entre le nombre d'angles et la valeur de ce paramètre traduit des quadratures bien connues de type Gauss-Legendre\cite{abbott_tricks_2005}. Cette équivalence est disponible dans le tableau \ref{tab:mmax};

\begin{table}[ht]
 \centering
  \begin{tabular}{l | c}
    \hline \multicolumn{2}{c}{} \\[-1em]\hline
    $\mathrm{m}_\mathrm{max}$ & nombre d'orientations \\
    \hline
    1  & 18 \\
    2  & 75 \\
    3  & 196 \\
    4  & 405 \\
    5  & 726 \\
    \hline \multicolumn{2}{c}{} \\[-1em]\hline
  \end{tabular}
  \caption[\'Equivalence entre le paramètre $\mathrm{m}_\mathrm{max}$ et le nombre d'angles.]{\'Equivalence entre le paramètre $\mathrm{m}_\mathrm{max}$ et le nombre d'angles considérés lors de la minimisation.}
  \label{tab:mmax}  
\end{table}


%Nous obtenons ainsi la fonctionnelle suivante:
%\begin{eqnarray}
%\mathcal{F} =& &\mathrm{k_B}T \sum\limits_{i=1}^{N_v}\mathrm{V}_i\sum\limits_{j=1}^{N_o}\mathrm{w}_{\Omega_j} \rho\left(\boldsymbol{r}_i,\Omega_j\right)\ln\left(\frac{\rho\left(\boldsymbol{r}_i,\Omega_j\right)}{\rho_0}\right)-\Delta\rho\left(\boldsymbol{r}_i,\Omega_j\right)\\
%			+& &\sum\limits_{i=1}^{N_v}\mathrm{V}_i\sum\limits_{j=1}^{N_o}\mathrm{w}_{\Omega_j} \rho\left(\boldsymbol{r}_i,\Omega_j\right)\phi\left(\boldsymbol{r}_i,\Omega_j\right) \nonumber \\
%            -& &\frac{\mathrm{k_B}T}{2}\sum\limits_{i=1}^{N_v}\mathrm{V}_i\sum\limits_{j=1}^{N_o}\mathrm{w}_{\Omega_j} \Delta\rho\left(\boldsymbol{r}_i,\Omega_j\right) \sum\limits_{k=1}^{N_v}\mathrm{V}_k\sum\limits_{l=1}^{N_o} \mathrm{w}_{\Omega_l} \nonumber \\
%            & &c\left(\boldsymbol{r}_i-\boldsymbol{r}_k,\Omega_j,\Omega_l \right) \Delta\rho\left(\boldsymbol{r}_k,\Omega_l\right) \nonumber 
%\end{eqnarray}
%\noindent avec $N_v$ le nombre de voxels dans le système, $N_o$ le nombre d'orientations considérées, $\mathrm{V}_i$ le volume du voxel d'indice i et $\mathrm{w}_{\Omega_j}$ le poids de l'orientation $\Omega_j$.


\subsection{Les convolutions}
Un des avantages majeurs de MDFT par rapport aux autres méthodes est sa rapidité. La partie idéale et la partie d'excès sont locales ce qui rend leur temps de calcul linéaire, proportionnel à $N_O N_V$. La partie qui nécessite le plus de temps et qui est donc limitante dans ce calcul est la partie d'excès qui est, elle, non locale. Afin de diminuer fortement le temps de calcul de cette partie et par conséquence le temps de calcul global, nous utilisons la propriété suivante des convolutions:
\begin{eqnarray}
f*g = \mathrm{FT}^{-1} [ \mathrm{FT}(f) . \mathrm{FT}(g) ]
\end{eqnarray}
La convolution de deux fonctions peut être calculée comme la transformée de Fourier inverse du produit point à point de la transformée de Fourier de ces deux fonctions. Pour rappel, la fonction $\gamma$ est la convolution entre les fonctions $\Delta\rho$ et $c$. Cette méthode est donc applicable mais ne permet cependant pas à elle seule de diminuer le temps de calcul. Cette propriété a donc été couplée à l'utilisation des FFT (Fast Fourier Transform) et en particulier de la librairie FFTW3 pour la partie spatiale et de FGSHT (fast generalized spherical harmonic transform) récemment proposé par Ding et al\cite{ding_thesis} pour la partie angulaire.
Les auteurs\cite{ding_thesis} ont montré que le couplage de ces deux méthodes permet d'obtenir des temps de calcul du même ordre de grandeurs pour chacune des 3 parties de la fonctionnelle. 


\subsection{Le minimiseur}
Le minimiseur utilisé pour minimiser notre fonctionnelle est L-BFGS (Limited-memory Broyden-Fletcher-Goldfarb-Shanno)\cite{Byrd_lbfgs_1995}. L-BFGS correspond à une version de BFGS\cite{bfgs_2006} optimisée pour les problèmes composés de nombreuses variables comme c'est le cas de la MDFT. Contrairement à BFGS qui conserve une approximation de la hessienne sous la forme d'une matrice dense, L-BFGS conserve uniquement quelques vecteurs représentatifs ainsi que l'historique sur quelques pas de la minimisation. L-BFGS nécessite en entrée, l'ensemble des variables à minimiser ainsi que le gradient de la fonctionnelle $\nabla F[\rho\left(\boldsymbol{r}_i,\Omega_j\right)]$ défini comme:
\begin{eqnarray}
\nabla F[\rho(\boldsymbol{r}_i,\Omega_j)] &=& \mathrm{k_B}T \ln(\frac{\rho(\boldsymbol{r}_i,\Omega_j)}{\rho_0}) \\
&+& \phi(\boldsymbol{r}_i,\Omega_j) \nonumber \\
&-& \mathrm{k_B}T \gamma(\boldsymbol{r}_i,\Omega_j) \nonumber
\end{eqnarray}
Le calcul du gradient de chaque partie est détaillé en annexe \ref{chap:annexes:grad}.

\section{Au-delà de HNC}
Pour aller plus loin que la théorie HNC, et ainsi corriger l'approximation faite dans la fonctionnelle d'excès, il est possible (i) d'appliquer des corrections à posteriori ou (ii) d'approximer le terme $\mathcal{O}(\Delta\rho^{3})$ via l'ajout d'un quatrième terme à notre fonctionnelle, une fonctionnelle de bridge.


\subsection{Les corrections à posteriori}
Les corrections à posteriori interviennent après la minimisation. Elles permettent donc uniquement de corriger la valeur de l'énergie libre de solvatation mais n'ont aucun impact sur la carte de densité du solvant. Des corrections de différents types ont été développées et sont actuellement utilisées dans la MDFT.


\subsubsection{Les corrections de pression}
Parmi ces corrections, deux permettent de corriger la pression du système. Pour rappel, l'approximation HNC correspond au premier ordre du développement de Taylor autour de la densité liquide. La phase gazeuse du solvant n'est donc pas représentée, ce qui entraîne une forte surestimation de la pression du système soit 10 000 bar. Nous décrivons de façon détaillé ce problème dans le chapitre \ref{chap:bridge}. Sergiievskyi et al. \cite{sergiievskyi_solvation_2015,sergiievskyi_pressure_2015} ont proposé une correction ad-hoc rigoureuse basée sur la théorie des liquides: la correction \textit{PC}. Au moment de ce développement, la théorie MDFT n'était pas encore au niveau HNC. Elle correspondrait aujourd'hui à une approximation de HNC avec $\mathrm{m}_\mathrm{max}$=1. Les auteurs ont de ce fait également proposé une correction empirique, \textit{PC+}, qui améliorait les résultats\cite{misin_salting-out_2016, misin_hydration_2016, misin_communication:_2015}. Nous montrerons dans le chapitre \ref{chap:BDD} que la correction \textit{PC+} n'est plus adaptée à la théorie dans l'approximation HNC. Nous proposerons également une alternative à ces corrections dans le chapitre \ref{chap:bridge} sous la forme d'un bridge gros gain.


% \subsubsection{Les corrections électrostatiques}
% VOIR AVEC MAX


\subsection{Les fonctionnelles de bridge}
Si l'on veut corriger à la fois l'énergie libre de solvatation et la densité du solvant, il est nécessaire de modifier la fonctionnelle. Pour cela, un quatrième terme nommé fonctionnelle de bridge est introduit dans la partie d'excès. Nous obtenons ainsi:
\begin{eqnarray}
\mathcal{F}_\mathrm{exc} = \mathcal{F}_\mathrm{exc}^{HNC} + \mathcal{F}_\mathrm{b}
\end{eqnarray}
Différentes formes pour la fonctionnelle de bridge ont été proposées ces dernières années\cite{levesque_scalar_2012,jeanmairet_molecular_2013,jeanmairet_molecular_2015}. Malheureusement, comme il a été montré dans un papier à venir, aucun de ces bridges ne permet une représentation du système totalement satisfaisante du point de vue thermodynamique. Dans la suite de ce rapport nous proposerons un nouveau bridge qui autorise la création d'une phase gazeuse de l'eau et permet ainsi de représenter de façon correcte la tension de surface de l'eau ainsi que la pression du système.



\clearpage
\strut
\vspace{10\baselineskip}

\boitemagique{A retenir}{
Danc ce chapitre, nous présentons la théorie de la fonctionnelle de la densité moléculaire.
Les récents développements de Ding et al ont permis de porter cette théorie au niveau de l'approximation HNC.
Nous présentons également les différentes corrections associées à cette théorie.
Malheureusement, aucune de ces corrections ne permet de reproduire l'ensemble des propriétés thermodynamiques de nos systèmes.
Dans le chapitre suivant, nous présentons une nouvelle correction adaptée aux systèmes macromoléculaires.
}


%Afin de corriger cela, plusieurs approximations de la fonctionnelle de bridge ont été proposées, des bridges compliqués et lourds qui permettent
%de retrouver de la consistante thermodynamique\cite{jeanmairet_molecular_2015} au prix de rdf très moches comme le bridge de sphères dures\cite{liu_bridge-functional-based_2014,levesque_scalar_2012},
%les bridges three body \cite{jeanmairet_molecular_2015}, etc.  Malheureusement, comme il a été montré dans un article à paraître, on ne peut pas construire un terme de bridge sphères dures qui soit
%complètement consistent, c'est à dire que l'on ai à la fois la pression du système, la tension de surface du solvant, tout en conservant des profils de densité et des énergies libres de solvatation corrects. 
