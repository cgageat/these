\chapter{Base de données}

\boitemagique{Objectif}{}


Une intro sur les bdd et leur buts


\section{Les bases de données}
Afin d'étudier la précision de MDFT, des bases de données de solvatation ont été étudiées.
\subsection{FreeSolv}
REPRENDRE LE RAPPORT DE JOSE ICI
La première base de données utilisée est FreeSolv [mobley et al]. Cette base de données est composé de 643 petites molécules accompagnées de leur énergie libre de solvatation issues de la littérature.
\subsubsection{Calculs de dynamiques moléculaires}
La base de données est également accompagnée de résultats de dynamiques moléculaires. Les paramètres de la simulations sont GAFF small molecule force field in TIP3P water with AM1-BCC charges. Les simulations ont été effectuées avec Gromacs.
Les paramètres issus de ces simulations sont:
\begin{itemize}
\item Un identifiant unique sous la forme mobley\_X ( X étant un nombre entre XXXXX et XXXXXX)
\item d\_h\_conf: 0.022435770142286173,
\item d\_expt: l'incertitude expérimentale en kcal/mol
\item iupac: Le nom iupac du composé
\item calc\_s: S calculé en cal/mol.K
\item d\_vdw: l'incertitude expérimentale en kcal/mol
\item d\_calc: 0.017,
\item d\_charging: 0.011,
\item calc\_vdw: L'énergie libre de solvatation charges partielles annulées (création de la cavité) en kcal/mol
\item groups: Les groupes chimiques présents dans la molécule parmis XXXXXXXXXXXXXXX
\item PubChemID: L'identifiant du composé dans la base de données pubchem. (ex: La molécule 887, sera accessible via l'url https://pubchem.ncbi.nlm.nih.gov/compound/887)
\item smiles: Le smile du composé
\item d\_calc\_s: en cal/mol/K
\item expt\_s: en cal/K/mol
\item calc\_h: H caclculé par DM
\item notes: Des notes sur le composé
\item nickname: Son surnom
\item h\_conf: 0.028144599471263517,
\item expt\_h: "Not available",
\item expt: L'énergie libre de solvatation expérimentale en kcal/mol
\item d\_calc\_h: L'incertitude sur le calcul du h en kcal/mol
\item expt\_reference: Le DOI de l'article duquel sont issu les valeurs expérimentales (ex: le DOI 10.1021/ct050097l, est accesible via l'url http://dx.doi.org/10.1021/ct050097l
\item d\_expt\_s: L'incertitude experimentale du calcul de S en cal/K.mol
\item calc: L'énergie libre de solvatation calculé par DM en kcal/mol
\item h\_solv: Le h solv en kcal/mol
\item d\_h\_solv: L'incertitude sur le calcul du h solv en kcal/mol
\item d\_expt\_h: l'incertitude sur le h exp en kcal/mol
\item calc\_charging: L'energie nécessaire à l'activation des charges une fois la cavité en kcal/mol
\item calc\_reference: Le DOI de l'article duquel sont issu les valeurs calculées (ex: le DOI 10.1021/ct050097l, est accesible via l'url http://dx.doi.org/10.1021/ct050097l
\item expt\_h\_reference: Le DOI de l'article duquel sont issu les valeurs de h expérimentales (ex: le DOI 10.1021/ct050097l, est accesible via l'url http://dx.doi.org/10.1021/ct050097l
\end{itemize}



\subsubsection{}
\subsubsection{}
\subsubsection{}
\section{MDT: the MDFT Database Tool}
REPRENDRE LE RAPPORT DE JOSE ICI


\section{Résultats}
REPRENDRE LE RAPPORT DE JOSE ICI
