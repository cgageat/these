\chapter{Développements numériques}
\label{chap:numerique}


\boitemagique{Objectif}{Dans ce chapitre, nous décrivons les développements numériques ainsi que les optimisations nécessaires à l'application de MDFT sur des systèmes biologiques.}

Afin de porter MDFT vers des applications biologiques, certains développements numériques ont été nécessaires. En effet ces systèmes imposent, de par leur taille, de nouvelles contraintes (taille de la boîte de simulation, mémoire nécessaire, ...) et rendent certaines parties du calcul bloquantes alors qu'elles étaient jusque là négligeables. C'est par exemple le cas avec le calcul du potentiel extérieur. Dans ce chapitre nous décrivons dans un premier temps le processus JUBE que nous avons mis en place afin de suivre et évaluer les différentes évolutions. Nous faisons ensuite une revue non exhaustive des améliorations les plus importantes. Le développement haute performance (HPC) est un aspect important de cette thèse. Le choix a cependant été fait de ne pas expliciter chaque terme de ce chapitre. Ces aspects sont décrits plus en détails dans les rapports EoCoE de MDFT.


\section{Reproductibilité}
La reproductibilité est une problématique récurrente lors de développement, de la modification ou de l'optimisation de logiciels de calcul. En effet, la comparaison de deux mesures liées à l’exécution d'un logiciel n'est pas pertinente si nous n'avons pas la certitude que les deux exécutions ont eu lieu strictement dans les mêmes conditions: options de compilation, environnement, options d’exécution, etc (voir image \ref{fig:JUBE_process}).

\begin{figure}[ht]
  \center
  \begin{tikzpicture}
  
    \tikzstyle{files}=[ellipse,draw,text=black]
    \tikzstyle{instruct}=[rectangle,draw,fill=yellow!50]
    \tikzstyle{test}=[diamond, aspect=2.5,thick,draw=blue,fill=yellow!50,text=blue]
    \tikzstyle{es}=[rectangle,draw,rounded corners=4pt,fill=blue!25]
    \tikzstyle{suite}=[->,>=stealth,line width=2pt,rounded corners=4pt]

    \node[files] (debut) at (0,0) {Sources};
    \node[es] (compilationOptions) at (5,-2) {Options de compilation};
    \node[instruct] (compilation) at (0,-2) {Compilation};
    \node[es] (executionOptions) at (5,-4) {Options d'exécution};
    \node[es] (softLink) at (-5,-4) {Logiciels d'analyses};
    \node[instruct] (execution) at (0,-4) {Exécution};
    \node[instruct] (analyse) at (0,-6) {Analyses};
    \node[files] (resultats) at (0,-8) {Résultats};
 
    \draw[suite] (debut) -- (compilation);
    \draw[suite] (compilationOptions) -- (compilation);
    \draw[suite] (compilation) -- (execution);
    \draw[suite] (executionOptions) -- (execution);
    \draw[suite] (softLink) -- (execution);
    \draw[suite] (execution) -- (analyse);
    \draw[suite] (analyse) -- (resultats);

  \end{tikzpicture}
  \caption{Processus d'exécution d'un logiciel de la récupération des sources à l'analyse des résultats.}
  \label{fig:JUBE_process}
\end{figure}

Afin de contrôler cette chaîne d’exécution, et de pouvoir relancer le même calcul des semaines, des mois plus tard, nous avons mis en place l’outil de gestion de flux JUBE\cite{Luhrs_parallel_2016,Galonska_advances_2012}.

\subsection{JUBE}
JUBE est un logiciel écrit en python et développé au \textit{Jülich Supercomputing Centre} qui permet d'une part, d'automatiser toutes les étapes nécessaires au lancement de cas tests et à leurs analyses et, d'autre part, de conserver un historique des exécutions précédentes et des résultats obtenus. Il allie la robustesse nécessaire à la reproductibilité et la flexibilité permettant d'adapter les cas tests et les mesures aux évolutions de MDFT. Un ensemble d'options (entièrement automatisables) nous permet d'étudier différentes métriques de MDFT. En effet, après une modification, qu'il s'agisse d'un changement global d'algorithme ou de la plus minime des optimisations, nous souhaitons dans un premier temps nous assurer que les résultats scientifiques (\'energie libre de solvatation, ...) sont inchangés, puis nous souhaitons étudier l'évolution des comportements informatiques (temps d'exécution, quantité de mémoire utilisée, ...).

Ce script a plusieurs objectifs:


\subsubsection{Les cas tests}
Il doit permettre de lancer simplement un ou plusieurs des 3 cas tests suivants:

\begin{table}[ht]
  \begin{center}
    \begin{tabular}{c c c c c}
      \hline & \\[-1em]\hline
      Nom & solute & Nombre de points de grille & taille de la boîte (\AA) & $\mathrm{m}_\mathrm{max}$  \\
      \hline
      petit & Pyridine & 64 & 20 & 3  \\
      moyen & lysosyme & 128 & 25 & 3  \\
      grand & lysosyme & 256 & 20 & 5  \\
      \hline & \\[-1em]\hline%
    \end{tabular}
  \end{center}
  \caption{Récapitulatif des 3 cas tests accessibles via JUBE.}
  \label{tab:JUBE_bench_cases}  
\end{table}


\subsubsection{L'environnement}
JUBE permet également l’exécution de MDFT sur différentes machines en s'adaptant à leurs environnements spécifiques. En effet, contrairement à des ordinateurs personnels, les super-calculateurs disposent généralement d'un système de gestion de tâches comme SLURM ou encore d'un système de gestion de modules qui permettent de charger des logiciels ou librairies indispensables à la bonne compilation/exécution de MDFT (GFORTRAN, FFTW3, JUBE, ...).


\subsubsection{Les options de compilation}
Afin de ne pas être impacté par les calculs précédents, les sources sont récupérées et recopiées depuis le dépôt distant (github\footnote{\url{https://github.com/}}) et recompilées pour chaque nouveau calcul. Afin d'étudier l'évolution au cours des modifications, il est possible de spécifier la version via le numéro de commit et/ou la branche à utiliser.

Les options de compilation s'adaptent également aux mesures effectuées. Si l'on prend l'exemple du calcul du taux de vectorisation, deux calculs sont lancés. Le premier, avec l'option de compilation empêchant la vectorisation nous sert de référence, le second, sans cette option, permet l'évaluation du taux de vectorisation. Il en est de même pour les options d'exécution.


\subsubsection{Les options d'exécution}
Afin de ne pas avoir un nombre infini de cas, la majorité des options est gérée via les 3 cas tests décrits précédemment. Il existe cependant des options indépendantes du cas étudié comme le nombre de processeurs utilisés ou encore le nombre d’itérations du calcul de la fonctionnelle effectuées.

Il est également possible à cette étape de coupler MDFT aux logiciels d'analyses suivants:
\begin{itemize}
\item darshan
\item scorep
\item scalasca
\item papi
\item VTune
\item valgrind
\end{itemize}
Cette thèse n'étant pas orientée vers le HPC, nous n'entrerons pas dans le détail du rôle et de l’exécution de ces différents logiciels.


\subsubsection{Les grandeurs mesurées}
L'ensemble des métriques extraites grâce aux logiciels listés ci-dessus sont décrits ci-dessous:

\begin{itemize}
\item[$\bullet$] Les métriques globaux:
  \begin{itemize}
    \item \textbf{Temps d’exécution}: Le temps réel d'exécution est fourni par MDFT et exprimé en secondes. Il correspond au temps nécessaire à l'exécution de MDFT.
    \item $\mathbf{\Delta G_{\mathrm{solv}}}$ : L’énergie libre de solvatation prédite par MDFT et exprimée en $\mathrm{kJ.mol}^{-1}$.
    \item \textbf{Nombre d'itérations} : Correspond au nombre d'itérations nécessaires à MDFT pour minimiser le système étudié.
  \end{itemize}
  \vspace*{1.5ex}% 

\item[$\bullet$] Les métriques OpenMP:
  \begin{itemize}
  \item \textbf{Répartition de charge} : La répartition de la charge, exprimée en \%. Elle permet d'évaluer le déséquilibre entre les différents threads OpenMP.
  \item \textbf{Temps OpenMP} : Exprimé en seconde, il correspond à la durée passée dans les parties du code parallélisées en OpenMP.
  \item \textbf{Ratio OpenMP} : Exprimé en pourcentage, le ratio OpenMP correspond au rapport entre le temps OpenMP et le temps total d'exécution. Un code séquentiel, a un ratio de 0, alors qu'un code entièrement parallélisé en OpenMP aura le ratio maximum 1. 
  \end{itemize}
  \vspace*{1.5ex}% 

\item[$\bullet$] Les métriques liées à la mémoire:
  \begin{itemize}
  \item \textbf{Empreinte mémoire}: L'empreinte mémoire correspond à la quantité maximum de mémoire utilisée lors du calcul. Elle est exprimée en Go.
  \item \textbf{Intensité d'utilisation du cache}: L'intensité d'utilisation du cache est exprimée en \% et correspond à la fraction des données directement disponibles en cache. Lors de la création ou de l'utilisation d'une variable, celle ci est copiée de la RAM vers le cache. Le cache est une mémoire restreinte, proche des processeurs, ce qui en rend l'accès très rapide. Lorsque le cache arrive à saturation, les variables les plus anciennes en sont supprimées et seront accessibles uniquement dans la RAM. Lors d'un accès à une variable, le processus vérifie dans un premier temps si elle est toujours dans le cache. C'est à ce taux de succès que correspond l'intensité d'utilisation du cache. Un ratio important correspond à un accès mémoire plus rapide et donc à un temps d'exécution plus faible.
  \end{itemize}
  \vspace*{1.5ex}% 

\item[$\bullet$] Les métriques liées à l'utilisation des processeurs:
  \begin{itemize}
  \item \textbf{IPC}: L'IPC correspond au nombre d'instruction par cycle. Plus ce nombre est important et plus MDFT exploite la puissance fournie par le processeur.
  \item \textbf{Temps d'exécution sans vectorisation}: Temps d’exécution d'un calcul sans vectorisation exprimé en sec. La vectorisation est désactivée à l'aide de l'option -fno-tree-vectorize pour Gfortran ou des options -no-simd et -no-vec pour ICC. 
  \item \textbf{Efficacité de la vectorisation}: L'efficacité de la vectorisation est mesurée en calculant le ratio entre le temps d'exécution avec et sans vectorisation. Plus ce nombre est important et plus MDFT exploite la puissance fournie par le processeur.
  \end{itemize}
  \vspace*{1.5ex}% 

\end{itemize}

Le script JUBE décrit ci-dessus, nous permet ainsi de suivre et de quantifier les évolutions de MDFT tout en nous assurant une constance dans les résultats fournis.


\section{Optimisations}
Lors de l'exécution de MDFT sur des systèmes biologiques, certaines parties dépendants du nombre d'atomes du soluté sont apparues limitantes alors que leurs temps d'exécution étaient négligeables jusque là. Le script JUBE précédemment décrit nous a permis d'identifier facilement ces parties comme par exemple le module qui calcule les forces Lennard-Jones ou encore le minimiseur. En collaboration avec la Maison de La Simulation (CEA), MDFT a également été parallélisé en OpenMP.


\subsection{Le module Lennard Jones}
Lors de l’initialisation du système, et en particulier du calcul du $\mathrm{V}_\mathrm{ext}$, un module calcul l’interaction Lennard-Jones entre chaque atome du soluté et chaque atome d'eau pour chaque point de grille et chaque orientation. Dans sa version naïve, le nombre de calculs effectués par ce module est de $N_{\mathrm{voxels}}\mathrm{x}N_{\mathrm{orientations}}\mathrm{x}N_{\mathrm{atomes\ du\ solute}}\mathrm{x}N_{\mathrm{atomes\ du\ solvant}}$. Les systèmes biologiques, composés de plusieurs milliers d'atomes, nécessitent une grande boîte de simulation. La quantité de calcul de ce module croît dont très rapidement. L'optimisation de ce calcul a eu lieu en deux étapes décrites ci-dessous.

\subsubsection{Ajout d'une distance limite}
Comme on le voit sur la figure \ref{fig:lj}, le potentiel de Lennard-Jones tend rapidement vers 0. Il n'est donc pas nécessaire de calculer ce potentiel pour des molécules trop distantes. Nous avons donc ajouté une distance limite, configurable par l'utilisateur, au delà de laquelle le potentiel n'est plus calculé. Malheureusement, une distance limite simple, sphérique, ne permet qu'un gain limité car nous sommes toujours obligés de calculer la distance entre chaque atome du soluté et chaque atome d'eau pour chaque point de grille et chaque orientation pour ensuite la comparer à notre distance limite. Nous avons donc fait le choix de sous-espaces cubiques de tailles égales à la valeur de la limite. En échange de quelques calculs supplémentaires, dans les angles, il n'est plus nécessaire de tester chaque distance. En effet, une représentation cubique permet de limiter les boucles de calcul à cette zone.



\pgfmathdeclarefunction{lj}{2}{%
    \pgfmathparse{4*#1*((#2/x)^12-(#2/x)^6)}%
}

\begin{figure}[ht]
    \center    
  \begin{tikzpicture}
    \begin{axis}[
            xlabel= distance (\AA),
            ylabel= Potentiel de Lennard-Jones (kJ.mol$^{-1}$),
            xmin = 2.95, xmax = 10,
            ymin = -1, ymax = 2,
            no markers,
            legend style = {draw = none, cells={anchor=west}}
      ]
      \addplot[mark=none, black, very thick, domain=2.95:10, samples=400] {lj(0.65,3.166)};
      \addplot[mark=none, black, very thick, dashed] plot coordinates {
        (2.95,  0)
        (10,  0)
};
    \end{axis}
  \end{tikzpicture}
    \caption{Exemple du potentiel de Lennard-Jones entre deux atomes d'oxygènes de l'eau SPC/E.}
    \label{fig:lj}
\end{figure}


\subsubsection{Mise en cache}
Malgré la puissance actuelle des ordinateurs, le calcul de fonctions trigonométriques reste coûteux en temps. Dans une implémentation naïve, il est nécessaire de reconstruire la position de chaque atome du solvant, pour chaque atome du soluté, pour chaque point de grille et pour chaque orientation. Si l'on considère l'étude de l'une des plus petites protéines existantes, le lysosyme (1960 atomes), dans une boîte divisée en 64 points dans chaque direction et pour une valeur de $\mathrm{m}_\mathrm{max}$=1, il est nécessaire d'effectuer $64^3\mathrm{x}3\mathrm{x}18\mathrm{x}1960$ soit environ 27.7 milliards reconstructions de position. Nous avons donc stocké en mémoire l'ensemble des positions relatives de chaque atome de solvant par rapport au point de grille étudié pour chaque orientation. Pour le même système, le nombre de reconstructions est donc aujourd'hui de 3x18 soit seulement 54. Le nombre de calculs est ainsi divisé par plus de 500 millions.


\subsubsection{Performances}
Si l'on reprend l'exemple du lysosyme (1960 atomes), dans une boîte de 32 \AA\ de coté divisée en 64 points dans chaque direction et pour une valeur de $\mathrm{m}_\mathrm{max}$=1, avant optimisation le module Lennard-Jones était complété en 1 h 47 min. Grâce à l'ensemble de ces optimisations, le même calcul est aujourd'hui complété en moins de 6 sec. Ces optimisations ont permi de diviser le temps dédié à ce module par plus de 1000.


\subsection{Le minimiseur: steepest descent}
Comme nous l'avons décrit dans le chapitre \ref{chap:theorie}, le minimiseur nativement implémenté dans MDFT est L-BFGS. Ce minimiseur est le meilleur compromis pour des systèmes comportants de nombreuses variables comme c'est le cas avec MDFT. Cependant, le nombre de variables reste limité. En effet, il stocke en mémoire un tableau de taille $2mn + 5n + 11mm + 8m$ avec n le nombre de variables à minimiser et m le nombre de pas d'historique que l'on conserve. En fortran, les entiers sont codés sur 32 bits. Ils sont donc limités à un maximum de $2^{32}$ soit 2.29e$^9$. Si l'on réduit l'historique au minimum, soit 1, la taille du tableau est de $7n+16$. Notre minimiseur est donc limité aux systèmes de moins de $\frac{2.29e9-16}{7}$ soit 6.14$^{8}$ variables. 

Dans notre cas, le nombre de variables est égal au nombre de points de grille multiplié par le nombre d'orientations dans chaque direction, soit $N_g^3 \mathrm{x} N_o$ avec $N_g$ le nombre de points de grille dans chaque dimension et $N_o$ le nombre d'orientations. La taille des boîtes de simulation est donc limitée à $\sqrt[3]{\frac{2^{32}-7}{7N_o}}$. La taille de boîte maximale autorisée en fonction de la valeur de $\mathrm{m}_\mathrm{max}$ est décrite dans le tableau \ref{tab:taille_boîte_max}.



\begin{table}[ht]
 \centering
  \begin{tabular}{l | c | c}
    \hline \multicolumn{3}{c}{} \\[-1em]\hline
    $\mathrm{m}_\mathrm{max}$ & Nombre d'orientations & Nombre de points de grilles autorisé \\
    \hline
    1  & 18 & 324 \\
    2  & 75 & 201 \\
    3  & 196 & 146 \\
    4  & 405 & 114 \\
    5  & 726 & 94 \\
    \hline \multicolumn{3}{c}{} \\[-1em]\hline
  \end{tabular}
  \caption{Taille de boîte maximum autorisée par L-BFGS en fonction du paramètre $\mathrm{m}_\mathrm{max}$.}
  \label{tab:taille_boîte_max}  
\end{table}


Afin de dépasser ces limites, nous avons donc implémenté un nouveau minimiseur dans MDFT: le steepest descent. Si l'on reprend l'exemple du lysosyme (1960 atomes), dans une boîte de 64 \AA\ de coté divisée en 128 points dans chaque direction, les performances obtenues en fonction de la valeur de $\mathrm{m}_\mathrm{max}$ avec les deux minimiseurs sont regroupées dans le tableau \ref{tab:perf_minimiseurs}.




\begin{table}[ht]
 \centering
  \begin{tabular}{c || c | c | c || c | c | c}
    \hline \multicolumn{3}{c}{} \\[-1em]\hline
         & \multicolumn{3}{c ||}{Mémoire utilisée (Go)} & \multicolumn{3}{c}{Temps de calcul} \\
    \hline
      $\mathrm{m}_\mathrm{max}$ & L-BFGS & SD & gain(\%) & L-BFGS & SD & gain(\%) \\
    \hline
    1  &  1.36 &  0.77 & 43.4 &  2 min 42 &  2 min 24 & 11.1 \\
    2  &  8.71 &  4.29 & 50.7 &  8 min 47 &  8 min 18 &  5.5 \\
    3  & 16.20 &  7.94 & 51.0 & 19 min 50 & 14 min 56 & 24.7 \\
    4  &   /   & 20.38 &  /   &    /      & 18 min 37 &   /  \\
    5  &   /   & 30.07 &  /   &    /      & 22 min 04 &   /  \\
    \hline \multicolumn{3}{c}{} \\[-1em]\hline
  \end{tabular}
  \caption[Comparaison des performances des minimiseurs L-BFGS et \textit{steepest descent}.]{Comparaison des performances des minimiseurs L-BFGS et \textit{steepest descent} dans le cas de la solvatation du lysosyme.}
  \label{tab:perf_minimiseurs}  
\end{table}

Comme on le voit dans ce tableau, en plus de rendre possibles les calculs les plus importants, le temps de calcul nécessaire au \textit{steepest descent} est légèrement inférieur à celui nécessaire à L-BFGS. Cela illuste l'importance de l'optimisation des accès mémoire. De plus, la mémoire utilisée par cette version de \textit{steepest descent} est moitié moins importante que celle utilisée par L-BFGS.




\subsection{La parallélisation OpenMP}
Afin d'améliorer les performances de MDFT et de bénéficier au maximum des architectures actuelles, les boucles les plus coûteuses en temps de calcul ont été parallélisées en OpenMP. Ce travail a été effectué en collaboration avec Yacine Ould-Rouis à la Maison de La Simulation. 

Les deux cadres d'utilisation de MDFT les plus coûteux sont:
\begin{itemize}
\item l'étude de bases de données complètes
\item l'étude d'une macro-molécule
\end{itemize}

Dans le premier cas, le nombre de calcul à lancer peut être largement supérieur au nombre de cœurs disponibles. La parallélisation n'a donc aucun avantage. En effet, à cause de la communication entre les différents processus, le temps de calcul nécessaire pour l’exécution d'un code parrallèle sur n cœurs est toujours supérieur au temps sur un cœur divisé par n. Dans ce cas de figure, il est donc plus intéressant de lancer chaque calcul en série soit sur un seul cœur.

Dans le second cas, c'est le temps de restitution qui est important. La parallélisation permet donc ici un gain de temps considérable. Les temps de calcul en fonction du nombre de cœurs OpenMP sont disponibles dans le tableau \ref{tab:perf_openmp}. Le système étudié est le lysosyme dans une boîte de simulation de 64 \AA\ de coté divisée en 128 points de grille dans chaque direction pour une valeur de $\mathrm{m}_\mathrm{max}$=3.



\begin{table}[ht]
 \centering
  \begin{tabular}{c | c | c}
      Nombre de cœurs & Temps de calcul & gain(\%)\\
    \hline
    sans OpenMP (ref) & 25 min & 0\\
     1 & 31 min 44 & -27.0 \\
     2 & 16 min 34 &  33.4 \\
     4 &  9 min 32 &  61.8 \\
     8 &  5 min 30 &  78.0 \\
    12 &  4 min 03 &  83.8 \\
    24 &  2 min 34 &  89.8 \\
    \hline \multicolumn{3}{c}{} \\[-1em]\hline
  \end{tabular}
  \caption[Temps de calcul en fonction du nombre de cœurs OpenMP.]{Temps de calcul en fonction du nombre de cœurs OpenMP. Le système étudié est le lysosyme dans une boîte de simulation de 64 \AA\ de coté divisée en 128 points de grille dans chaque direction pour une valeur de $\mathrm{m}_\mathrm{max}$=3.}
  \label{tab:perf_openmp}  
\end{table}


On voit que sur un coeur, l'éxecution de MDFT est plus lente avec OpenMP (31min) que sans (25min). Ces résultats montrent que l'utilisation de OpenMP est co\^uteuse en temps de calcul. C'est pourquoi, dans le cas de benchmark, il est plus efficace de lancer les calculs en série. On voit également que l'utilisation de 24 cœurs permet de faire passer le temps de calcul de 25 min à seulement 2 min 34 et ainsi de le diviser par 10. Il aurait bien sûr été possible d'encore plus optimiser la parallélisation de MDFT, cependant le choix a été fait de s’arrêter ici afin que le code ne perde pas en lisibilité.



% \subsubsection{Identification des hot spots}
% La première étape nécessaire à toute optimisation est l'analyse des boucles les plus longue, appelées hot spots. Pour cela, nous nous sommes servi de l’outil de profilage VTune[ref VTune]. Cet outils permet une analyse HPC complète.

% \subsubsection{Modification du format de données}

\clearpage
\strut
\vspace{10\baselineskip}

\boitemagique{A retenir}{Dans ce chapitre, nous avons dans un premier décrit la mise en place d'un outil permettant un suivi simple et efficace de l'évolution de MDFT. Nous avons ensuite présenté les développements numériques et optimisations qui permettent aujourd'hui d'atteindre des tailles de systèmes intéressantes en biologie dans des temps de calcul raisonnables.}
