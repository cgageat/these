\section{gradients de la fonctionnelle} \label{app:grad}

\begin{eqnarray}
\widetilde{\rho}(\vec{k})&=&HT[\rho(\vec{r})]  \\
\bar{\widetilde{\rho}}(\vec{k})&=&\widetilde{\rho}(\vec{k})K(\vec{k})   \\
\bar{\rho}(\vec{r})&=&HT^{-1}[\bar{\widetilde{\rho}}(\vec{k})]  \\
\Delta\bar{\rho}(\vec{r})&=&\bar{\rho}(\vec{r})-\rho_{0}  \\
\bar{F}_{b}[\rho(\vec{r})]&=&A\Delta\bar{\rho}(\vec{r})^{3}+B\bar{\rho}(\vec{r})^{2}\Delta\bar{\rho}(\vec{r})^{4}  \\
\frac{d\bar{F}_{b}[\rho(\vec{r})]}{d\rho(\vec{r})}&=&3A\Delta\bar{\rho}(\vec{r})^{2}+2B\bar{\rho}(\vec{r})\Delta\bar{\rho}(\vec{r})^{4}+4B\bar{\rho}(\vec{r})^{2}\Delta\bar{\rho}(\vec{r})^{3}  \\
\frac{d\bar{\widetilde{F}}_{b}[\rho(\vec{r})]}{d\rho(\vec{r})}&=&HT[\frac{d\bar{F}_{b}[\rho(\vec{r})]}{d\rho(\vec{r})}]  \\
\frac{d\widetilde{F}_{b}[\rho(\vec{r})]}{d\rho(\vec{r})}&=&\frac{d\bar{\widetilde{F}}_{b}[\rho(\vec{r})]}{d\bar{\rho}(\vec{r})}K(\vec{k})  \\
\frac{dF_{b}[\rho(\vec{r})]}{d\rho(\vec{r})}&=&HT^{-1}\frac{d\widetilde{F}_{b}[\rho(\vec{r})]}{d\rho(\vec{r})} 
\end{eqnarray}



\section{Etude paramètrique du bridge} \label{app:etudeParam}
%\begin{figure}[!htbp]
%    \center
%    \includegraphics[width=8cm]{fonctionnelles.eps}
%    \caption{Free energy of homogeneous solvent of density $\rho$. $\rho_b$ is the bulk density of SPC/E water in room condition (1$kg.L^{-1}$) in the HRF approximation. }
%    \label{fig:fonctionelle}
%\end{figure}

\begin{figure}[!htbp]
  \begin{tikzpicture}
    \begin{axis}[
            xlabel= r ($\text{\AA}$),
            ylabel= g(r),
            xmin = 0, xmax = 1.5,
            ymin = 0, ymax = 0.15,
            legend style = {draw = none, cells={anchor=west}}
      ]
      \addplot+[mark=none, very thick] file {datas/fonctionnelles/fonctionnelle_-15.0.csv};
      \addplot+[mark=none, very thick] file {datas/fonctionnelles/fonctionnelle_-10.0.csv};
      \addplot+[mark=none, very thick] file {datas/fonctionnelles/fonctionnelle_-5.0.csv};
      \addplot+[mark=none, very thick] file {datas/fonctionnelles/fonctionnelle_0.0.csv};
      \addplot+[mark=none, very thick] file {datas/fonctionnelles/fonctionnelle_5.0.csv};
      \addplot+[mark=none, very thick] file {datas/fonctionnelles/fonctionnelle_10.0.csv};
      \addplot+[mark=none, very thick] file {datas/fonctionnelles/fonctionnelle_15.0.csv};
      \legend{$-15.10^8$, $-10.10^8$, $-5.10^8$, $0.10^8$, $5.10^8$, $10.10^8$, $15.10^8$}
    \end{axis}
  \end{tikzpicture}
      \caption{Free energy of homogeneous solvent of density $\rho$. $\rho_b$ is the bulk density of SPC/E water in room condition (1$kg.L^{-1}$) in the HRF approximation. }
    \label{fig:fonctionelle_bench}
\end{figure}





\section{courbe présentant l'effet de la largeur du kernel sur un g(r). Plus le kernel est large et plus la densité sera lissée, comme attendu.} \label{app:curve}
\begin{figure}[!htbp]
    \center
      \begin{tikzpicture}
    \begin{axis}[
            xlabel= r ($\text{\AA}$),
            ylabel= g(r),
            xmin = 0, xmax = 12,
            ymin = 0, ymax = 2,
            legend style = {draw = none, cells={anchor=west}}
      ]
      \addplot+[mark=none, very thick] file {datas/g_of_r/bridge_1.177_-15/g_methane.csv};
      \addplot+[mark=none, very thick] file {datas/test_kernel_size/cgG_0.500000_.csv};
      \addplot+[mark=none, very thick] file {datas/test_kernel_size/cgG_1.000000_.csv};
      \addplot+[mark=none, very thick] file {datas/test_kernel_size/cgG_1.500000_.csv};
      \addplot+[mark=none, very thick] file {datas/test_kernel_size/cgG_2.000000_.csv};
      \addplot+[mark=none, very thick] file {datas/test_kernel_size/cgG_100.000000_.csv};
      \legend{0.0, 0.5, 1.0, 1.5, 2.0, 100.0}
    \end{axis}
  \end{tikzpicture}
    \caption{Coarse grained radial distribution function of water around unified methane with different value for the parameter $\sigma$ of the kernel}
    \label{fig:coarse_grained_effect}
\end{figure}







\section{Temps de calculs} \label{app:calculTime}
La version de la théorie à symmétrie radiale décrite ci-dessus, à été implémentée dans un petit logiciel en C++. Ce logiciel prend en entrée le rayon du système à étudier ainsi que le nombre de point de grille. Notre système étant à symmétrie radiale, chaque point de grille correspond en réalité à une coquille. Les temps nécessaires à la minimisation de la functionnelle et à la production des profils de densité ainsi que de l'énergie libre de solvatation en fonction du nombre de points du système sont disponibles dans le tableau \ref{tab:times}.

\begin{table}[!htbp]
  \begin{tabular}{ l c c}
    \hline & \\[-1em]\hline
    N   & HRF  & HRF + bridge \\
    \hline
    100     & 0.36 & 0.50 \\
    200     & 0.68 & 1.05 \\
    500     & 1.63 & 2.33 \\
   1000     & 2.99 & 4.86 \\
   2000     & 5.81 & 9.10 \\
    \hline & \\[-1em]\hline
  \end{tabular}
  \caption{Temps en seconds nécessaire pour générer le profil de solvatation et l'énergie libre de solvatation avec la version à gémoétrie sphérique de MDFT en fonction du nombre de points}
  \label{tab:times}
\end{table}

\section{Paramètres des calculs} \label{app:calculParam}
Les calculs ont été effectués dans des systèmes de rayon 15$\text{\AA}$ pour les gaz rares, le méthane et le néopentane et de 120 $\text{\AA}$ pour les sphéres dures avec un espacement entre deux points de 0.1$\text{\AA}$
Les paramètres électrostatiques utilisés sont disponibles dans le tableau \ref{tab:electrostatParam}.

\begin{table}[!htbp]
  \begin{tabular}{ l c c }
    \hline & \\[-1em]\hline
    compound   & $\sigma (\text{\AA}^{-1})$  & $\epsilon (kJ.mol^{-1})$ \\
    \hline
    Methane    & 3.730 & 1.23000 \\
    Neopentane & 6.150 & 3.49600 \\ 
    Neon       & 3.035 & 0.15432 \\
    Argon      & 3.415 & 1.03931 \\
    Krypton    & 3.675 & 1.40510 \\
    Xenon      & 3.975 & 1.78510 \\
    \hline & \\[-1em]\hline
  \end{tabular}
  \caption{Electrostatic parameters of molecules used in MDFT calculations.}
  \label{tab:electrostatParam}  
\end{table}


\section{Tecnhical tips and tricks} \label{app:technical_tips_adn_tricks}
Afin d'accélerer considérablement le temps de calcul, le calcul de la partie d'excès et en particulier de la convolution se fait 

Je peux parler de:
- transformée de fourier => transformée de hankel
- mise en cache des tarnsformées de hankel car extrémement répétitif

Pour éviter que la fonctionnelle diverge dans le cas de large soluté, le nombre de point dans l'espace inverse à été fortement augmenté. L'espace inverse comporte un point tout les 0.01$\text{\AA}^{-1}$ sur 250 $\text{\AA}^{-1}$. Nous pouvons certainement diminuer ce nombre de points sans perdre en précision mais les temps de calculs étant très court gràçe à nos astuces précédentes, nous avons fait le choix de les laisser.





\section{Distribution function for models molecules} \label{app:g_of_r}


\begin{figure}
    \begin{subfigure}[b]{0.25\textwidth}
        \centering
        \resizebox{\linewidth}{!}{
			%\plotG{argon}
        }
        \caption{argon}
    \end{subfigure}%
    \begin{subfigure}[b]{0.25\textwidth}
    \centering
        \resizebox{\linewidth}{!}{
            %\plotG{krypton}
        }
        \caption{krypton}
    \end{subfigure}
    \begin{subfigure}[b]{0.25\textwidth}
        \centering
        \resizebox{\linewidth}{!}{
            %\plotG{neon}
        }
        \caption{neon}
    \end{subfigure}%
    \begin{subfigure}[b]{0.25\textwidth}
        \centering
        \resizebox{\linewidth}{!}{
            %\plotG{xenon}
        }
        \caption{xenon}
    \end{subfigure}
    \begin{subfigure}[b]{0.25\textwidth}
        \centering
        \resizebox{\linewidth}{!}{
            %\plotG{methane}
        }
        \caption{methane}
    \end{subfigure}%
    \begin{subfigure}[b]{0.25\textwidth}
        \centering
        \resizebox{\linewidth}{!}{
            %\plotG{neopentane}
        }
        \caption{neopentane}
    \end{subfigure}
    \caption{Radial Distribution Function of rare gas (argon, neon, krypton, xenon) and two united molecules: methane and neopentane. Les pointillés correspondent à des simulations de dynamique moléculaires. RDF where calculated with MDFT with (in green) and without (in blue) our new bridge function. References where calculated by MD (dashed line).}
    \label{fig:g_of_r_complete}
\end{figure}

