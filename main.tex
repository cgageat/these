\documentclass{bredele}

\usepackage{thesis}

\title{Solvatation de systèmes d’intérêt pharmaceutique : apports de la théorie de la fonctionnelle de la densité moléculaire}

\author{Cédric Gageat}

\institute{l'\'Ecole Normale Supérieure}

\supervisor[Maximilien Levesque]{Daniel Borgis}
\doctoralschool{Chimie physique et chimie analytique de Paris Centre}{388}
% \specialty{Mathématiques, Informatique Temps-Réel, Robotique}
\date{24 Novembre 2017}

\jury{
  Mme Francesca Ingrosso\\
  Université de Lorraine, Rapportrice

  M. Thomas Simonson\\
  \'Ecole polytechnique, Rapporteur

  Mme Liliane Mouawad\\
  Institue Curie, Membre du jury

  M. Jean-Philip Piquemal\\
  UPMC, Membre du jury

  M. Ivan Duchemin\\
  CEA/INAC, Membre du jury

  M. Daniel Borgis\\
  \'Ecole Normale Supérieure, Directeur

  M. Maximilien Levesque\\
  \'Ecole Normale Supérieure, Encadrant
}

\frabstract{Les processus se déroulant à l'état liquide, comme les réactions chimiques, par exemple, se déroulent dans un océan de solvant. Peut­on simuler ou modéliser ces phénomènes en solution?
(i) Approximativement oui, en utilisant des méthodes approximatives reliées à une description macroscopique du solvant. Ces méthodes sont rapides (quelques secondes de calculs) mais ne sont pas capable des capturer la nature moléculaire ou les effets physique du solvant tels que les effets dus à l'orientation du solvant ou aux liaisons hydrogènes par exemple.
(ii) Oui, précisément en utilisant des techniques de simulations explicites comme la dynamique moléculaire. Mais elles ont au moins 3 ou 4 ordre de grandeurs de plus en lenteur. Des centaines, voir des milliers d'heures de temps de calcul sont généralement nécessaires.
(iii)  Nous allons présenter la théorie de la fonctionnelle de la densité et son code associé, MDFT. Nous allons montrer comment l'état de l'art de la théorie des liquides et les algorithmes haute performances, peuvent capturer les effets de solvatation à l'échelle moléculaire, avec un coût de calcul similaire aux méthodes approximatives.
}

\enabstract{Processes taking place in the liquid state, for instance chemical reactions, happen in a sea of solvent molecules. Can we model or predict their effect in solution?
(i) Roughly yes, using rough methods that rely on a macroscopic description of the solvent. They are fast (say few cpu-seconds) but are not able to capture the physical, molecular nature of solvation. No packing, no orientation effects, no hydrogen-bonding, among others.
(ii) Yes, accurately, using explicit simulations like molecular dynamics. But these are at least 3 to 4 orders of magnitude slower. Hundreds or thousands of cpu-hours are often necessary.
(iii) We will present the molecular density functional theory and its associated code, MDFT. We will show how state-of-the-art liquid state theory and high performance algorithms can capture solvation effects at their inherent molecular scale, for the cost of rough methods.
}

\frkeywords{solvatation biomolécules théorie de la fonctionnelle de la densité watermap.}
\enkeywords{solvation biomolecules Molecular density functional theory watermap}

\begin{document}

\frontmatter

\tikzexternaldisable
\maketitle{}
\tikzexternalenable

%\tikzexternaldisable
%\maketitle
%\tikzexternalenable

\cleardoublepage

\chapter*{Remerciements}
%\thispagestyle{empty}
%Blabla

%Ne pas oublier nicolas pour MM-PBSA et les molécules bio

\clearemptydoublepage



%\pagestyle{headings}
\renewcommand\contentsname{Sommaire}
\tableofcontents
 
  % **************** List of Tables, Illustrations etc *****************
 
\renewcommand{\cftdotsep}{\cftnodots}
\cftpagenumbersoff{figure}
\cftpagenumbersoff{table}
\cleardoublepage
\listoffigures
\cleardoublepage
\listoftables


\clearemptydoublepage
\section*{Notations}

\begin{tabular}{l l}
$\boldsymbol{r}$ & Position, en 3D, de la molécule d'eau étudiée \\
$\Omega$ & Orientation de la molécule d'eau étudiée \\
$\rho\left(\boldsymbol{r},\Omega \right)$ & Densité en solvant à la position $\boldsymbol{r}$ et pour l'orientation $\Omega$ [\AA$^{-3}$]  \\  
$\rho_0$ & Densité bulk de référence (1 kg.L$^{-1}$ soit 0.033 \AA$^{-3}$ pour l'eau) \\
$\mathcal{F}[\rho\left(\boldsymbol{r},\Omega \right)]$ & Fonctionnelle de la densité moléculaire $\rho$ \\
$\mathcal{F}_{id}[\rho\left(\boldsymbol{r},\Omega \right)]$ & Partie idéale de la fonctionnelle de la densité moléculaire [kJ.mol$^{-1}$]\\
$\mathcal{F}_{ext}[\rho\left(\boldsymbol{r},\Omega \right)]$ & Partie extérieure de la fonctionnelle de la densité moléculaire  [kJ.mol$^{-1}$]\\
$\mathcal{F}_{exc}[\rho\left(\boldsymbol{r},\Omega \right)]$ & Partie d'excès de la fonctionnelle de la densité moléculaire [kJ.mol$^{-1}$]\\
$\mathcal{F}_{b}[\rho\left(\boldsymbol{r},\Omega \right)]$ & Fonctionnelle de bridge [kJ.mol$^{-1}$]\\
$\phi\left(\boldsymbol{r},\Omega \right)$ & Potentiel d'interaction entre le soluté et le solvant à la position $\boldsymbol{r}$ et pour \\
 & l'orientation $\Omega$ [kJ.mol$^{-1}$]\\
$\mathrm{k_B}$ & Constante de Boltzmann. $\mathrm{k_B}$=8.3144598e$^{-3}$ [kJ.mol$^{-1}$.K$^{-1}$]\\
$c\left(\left|\boldsymbol{r}-\boldsymbol{r}^\prime\right|,\Omega,\Omega^\prime \right)$ & Fonction de corrélation directe entre la densité à la position $\boldsymbol{r}$ et pour \\
 & l'orientation $\Omega$ et la densité à la position $\boldsymbol{r}^\prime$ et pour l'orientation $\Omega^\prime$\\
$\gamma(\boldsymbol{r},\Omega)$ & Résultat de la convolution entre la  fonction de corrélation directe et la\\
& fonction $\Delta\rho$\\
$\gamma$ & Tension de surface [mJ.m$^{-2}$]\\
$\Delta G_{solv}$ & \'Energie libre de solvatation [kJ.mol$^{-1}$]\\
f$\ast$g & Convolution entre les fonctions f et g\\
$\hat{f}$ & Transformée de Fourier de la fonction f\\
$\boldsymbol{k}$ & Vecteur réciproque\\
$\omega$ & Orientation réciproque \\
$\bar{\rho(\boldsymbol{r})}$ & Densité gros grain à la position $\boldsymbol{r}$ [\AA$^{-3}$] \\
$\beta$ & Inverse du produit de la constante de Boltzmann et de la température \\
 & $(\mathrm{k_B}T)^{-1}$ [mol.kJ$^{-1}$]\\
\end{tabular}

\section*{Acronymes}
\begin{tabular}{ll}
MDFT & Théorie de la fonctionnelle de la densité moléculaire\\
HNC & Hypper-Netted Chain approximation\\
DM & Dynamique moléculaire \\
MC & Monte-Carlo \\
PDB & Protein data bank \\
FT & Transformée de Fourier \\
FFT & Transformée de Fourier rapide \\
HT & Transformée de Hankel \\
FGSHT & Transformée des harmoniques sphériques généralisées rapide \\
RDF & Fonction de distribution radiale\\
\end{tabular}







\clearemptydoublepage
\mainmatter

\part{Introduction et théorie}

%I    Introduction
\clearemptydoublepage
\import{chapters/introduction/}{content.tex}

%II   Théorie
\clearemptydoublepage
\import{chapters/theorie/}{content.tex}

\part{Développements théoriques}

%III  Bridge
\clearemptydoublepage
\import{chapters/bridge/}{content.tex}

\part{Développements numériques}

%IV   Numérique
\clearemptydoublepage
\import{chapters/numerique/}{content.tex}


%V    Base de données
\clearemptydoublepage
\import{chapters/BDD/}{content.tex}

\part{Applications}

%VI  Applications
\clearemptydoublepage
\import{chapters/Applications/}{content.tex}


\part{Conclusion et perspectives}


%VI  Conclusion
\clearemptydoublepage
\import{chapters/Conclusion/}{content.tex}



% ANNEXES
\clearemptydoublepage
\renewcommand{\thesubsection}{\Alph{chapter}}
\import{.}{annexes.tex}


\clearemptydoublepage
\backmatter
\printbibliography

\tikzexternaldisable

\end{document}
